 \documentclass{beamer}

\usetheme{MagdeburgFIN}
\usefonttheme{structurebold}
\usepackage{graphicx}
\usepackage{float}
\usepackage{url}
\usepackage{pdfpages}


\title{Build Your Own OctopusDB.\\ Blinktopus}
\author{Ali Hashaam, Ali Memon, Guzel Mussilova, Pavlo Shevchenko}
\date{May 2, 2017}
\institute{Scientific Project: Databases for Multi-Dimensional Data, Genomics and Modern Hardware}

\begin{document}

\begin{frame}[plain]
 \titlepage
\end{frame}

\begin{frame}
\frametitle{Table of Contents}
\tableofcontents
\end{frame}

\section{Introduction to the Topic}

\subsection{Motivation}
\begin{frame}
\frametitle{Motivation}
Modern enterprises need to pick the right DBMSs for their data managing problems.\\ \pause
\hspace{0.2 cm} 1. Use specialized solution for each application. \\ \pause
\hspace{0.5 cm} $\rightarrow$ costly due to licensing fees, integration overhead and DBA costs\\ \pause
\hspace{0.2 cm} 2. Use a single specialized DBMS for all applications. \\ \pause
\hspace{0.5 cm} $\rightarrow$ compromise heavily on performance.
\footnote{A. Jindal. The Mimicking Octopus: Towards a one-size-fits-all Database Architecture, 2010}
\end{frame}

\subsection{Idea of OctopusDB}
\begin{frame}
\frametitle{Idea of OctopusDB}
Create a new type of database system without fixed store that will mimic several existing systems. \\ \pause
\begin{itemize}
\item{Storage Views}
\end{itemize}
Like "real" octopuses can mimic other creatures and adjust to the environment
\end{frame}

\begin{frame}
\frametitle{Idea of OctopusDB}
\includegraphics[scale=0.5]{img/octopus_arch.png}
\footnote{A. Jindal. The Mimicking Octopus: Towards a one-size-fits-all Database Architecture, 2010}
\end{frame}

\subsection{Our Goal}
\begin{frame}
\frametitle{Our Goal}
\begin{itemize}
\item{Not to \textbf{clone} OctopusDB} \pause
\item{Provide a \textbf{framework} that gives user a chance to act as \textit{Holistic SV Optimizer}} \pause
\item{Add \textbf{Approximate Query Processing (AQP)}. BlinkDB}
\pause
\item{\textbf{Evaluate} performance depending on choice of SV}
\end{itemize}
\end{frame}

\section{Approximate Query Processing}
\subsection{Schedule}
\begin{frame}
\frametitle{AQP. Motivation}
The goal is to provide approximate answers with acceptable accuracy in orders of magnitude less time than that for the exact query processing.\footnote{Liu, Qing. Approximate Query Processing (Reference work entry) in: Liu, Ling, and M. Tamer Özsu. Encyclopedia of database systems. Vol. 6. Berlin, Heidelberg, Germany: Springer, 2009.}\\
\vspace{0.3 cm}
Revised query optimization goals:
\begin{itemize}
\item{Efficiently Accessing all query-relevant tuples.}
\item{Choose the best query plan among the available equivalent ones.}
\end{itemize}
In many cases, not all query-relevant data needs to be accessed.

\end{frame}

\subsection{Workflow}
\begin{frame}
\frametitle{AQP. Workflow}
\includegraphics[scale=0.5]{img/blinkdb-workflow.png}
\footnote{The general anatomy of approximate query processing system}
\end{frame}

\subsection{Synopsis Manager}
\begin{frame}
\frametitle{AQP. Synopsis Manager}
A synopsis captures essential properties of the real data while taking less space.\\
\vspace{0.3 cm}
The synopses manager is responsible for:
\begin{itemize}
\item{Type of summary to use(Samples, histograms, sketches, wavelets etc.)}
\item{When to build it (offline vs. online)}
\item{How to store it (to use overlapping samples, how to structure/index/cache the synopses)}
\item{When to update it (batch or online)}
\end{itemize}
\end{frame}

\subsection{Our Vision}
\begin{frame}
\frametitle{Our Vision}
\includegraphics[scale=0.45]{img/flow.jpeg}
\end{frame}

\section{Building a Blinktopus: What is our challenge?}
\begin{frame}
\frametitle{Building a Blinktopus: What is our challenge?}
First, the Octopus:
\begin{itemize}
\item{Store incoming data in logs.}
\item{Query the logs (just a filter query)}
\item{Allow users to create views (row, column) over certain logs.}
\item{List all views and logs}
\item{Launch the query over views or over logs, see the changes in performance.}
\end{itemize}
\end{frame}
\begin{frame}
\frametitle{Building a Blinktopus: What is our challenge?}
Enter AQP:
\begin{itemize}
\item{What synopsis can we easily support as a view for a specific query? Which will we choose to test? (Samples, histograms?)}
\item{Do Octopuses and AQP match well together?}
\item{How will we allow users to build this view?}
\item{How will we support queries using this view?}\\
\vspace{0.5 cm}
Blinktopuses are entirely a novel idea\\ \textbf{No one has done this before, ever..!}
\end{itemize}
\end{frame}

\section{Project Organisation}


\subsection{Roles}
\begin{frame}
\frametitle{Project Organisation.Roles}

\textbf{Team:} \\
\hspace{0.3 cm}Guzel - Manager (Team Leader)\\
\hspace{0.3 cm}Ali H. - Developer\\
\hspace{0.3 cm}Ali M. - Developer\\
\hspace{0.3 cm}Pavlo - Researcher\\

\vspace{0.2 cm}
\textbf{Supervisor:} \\
\hspace{0.3 cm} Gabriel Campero Durand \\

\vspace{0.2 cm}
Changing roles after each milestone.

\end{frame}


\subsection{Schedule}
\begin{frame}
\frametitle{Project Organisation.Schedule}
\textbf{Milestones} \\
\vspace{0.5 cm}
\begin{tabular}{|c|c|}
\hline 
02.05.2017 & MS-I (Kick-Off) \\ 
\hline 
23.05.2017 & MS-II (Concepts) \\ 
\hline 
13.06.2017 & MS-III (Implementation) \\ 
\hline 
04.07.2017 & MS-IV (Final) \\ 
\hline 
\end{tabular}
\\ \vspace{0.5 cm}
\textbf{Meetings} \\ 
\vspace{0.5 cm}
\hspace{0.5 cm} Team Meetings: Mo 14-15 \\
\hspace{0.5 cm} Meetings with supervisor: We 10-11

\end{frame}

\begin{frame}
 \frametitle{Thank you for your attention! Any questions?}
\end{frame}

\section{Literature}
\begin{frame}
\frametitle{Literature}
\begin{enumerate}
\item{Jindal, Alekh. "The mimicking octopus: Towards a one-size-fits-all database architecture." VLDB PhD Workshop. 2010.}
\item{Dittrich, Jens, and Alekh Jindal. "Towards a One Size Fits All Database Architecture." CIDR. 2011.}
\item{Jindal, Alekh. "OctopusDB: flexible and scalable storage management for arbitrary database engines." (2012).}
\item{Idreos, Stratos, Martin L. Kersten, and Stefan Manegold. "Database Cracking." In CIDR, vol. 7, pp. 68-78. 2007.}
\item{Mozafari, Barzan. "Approximate query engines: Commercial challenges and research opportunities." SIGMOD, 2017.}
\end{enumerate}
\end{frame}


\end{document}
